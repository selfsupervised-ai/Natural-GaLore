
\vspace{-4mm}
\section{Introduction}
Large Language Models (LLMs) have achieved remarkable performance across various disciplines, including conversational AI and language translation. However, training and fine-tuning these models demand enormous computational resources and are highly memory-intensive. This substantial memory requirement arises from storing billions of trainable parameters along with associated gradients and optimizer states.

To quantify this, consider a model with $\Psi$ parameters which is being trained using the Adam optimizer. In this case, storing parameters and their gradients in 16-bit precision formats like FP16 or BF16 requires $2\Psi$ bytes each. The associated optimizer states are typically stored in 32-bit precision (FP32) for numerical stability, necessitating an additional $4\Psi$ bytes for each parameter, gradient momentum, and variance, amounting to $12\Psi$ bytes. Therefore, the total memory requirement sums up to $16\Psi$ bytes. When accounting for model-dependent memory, such as activations during forward and backward passes, and residual memory, like temporary buffers and memory fragmentation, the overall memory footprint can easily exceed $18\Psi$ bytes \citep{raffelExploringLimitsTransfer2020,touvronLlamaOpenFoundation2023,chowdheryPaLMScalingLanguage2022}.

This enormous memory demand poses significant challenges, especially when training LLMs on hardware with limited memory capacity. As models continue to scale, efficient memory utilization becomes critical for making training feasible and accessible. In this work, we develop an efficient adaptation to the GaLore algorithm \citep{zhao2024galore}, which significantly reduces the memory footprint during training and fine-tuning of LLMs by approximating the optimizer state. Our approach, \textit{\lowrank}, leverages the low-rank structure of gradients and incorporates second-order information to achieve faster convergence and higher performance without additional memory overhead and can be used as a drop in replacement \footnote[1]{All code to reproduce the results are available at: https://github.com/selfsupervised-ai/Natural-GaLore.git} to standard optimization algorithms like Adam and AdamW.

\paragraph{Parallel and Distributed Training Techniques}
Researchers have developed various distributed computing techniques that leverage system-level optimizations and hardware resources to mitigate the substantial memory requirements in training LLMs.

One prominent framework is \textit{Distributed Data-Parallel (DDP)} that combines data parallelism where the training dataset is partitioned across multiple devices or nodes, with efficient gradient synchronization mechanisms, minimizing communication overhead. While data parallelism efficiently utilizes multiple GPUs, it can still face memory bottlenecks when model sizes exceed the memory capacity of a single device.

\textit{Model parallelism} addresses this limitation by partitioning the model across multiple devices, allowing for the training of models that are too large to fit into the memory of a single GPU. Techniques like \textit{pipeline parallelism} \citep{huangGPipeEfficientTraining2019} and \textit{tensor parallelism} \citep{shoeybiMegatronLMTuningScaling2019} enables the distribution of different layers or partitions of layers across devices. However, model parallelism introduces communication overhead and can be complex to implement effectively.

Another effective technique is \textit{gradient checkpointing} \citep{chenTrainingDeepNets2016}, which reduces memory usage by selectively storing only a subset of activations during the forward pass and recomputing them during the backward pass as needed. This approach trades increased computational overhead for reduced memory consumption, enabling the training of deeper models without exceeding memory constraints.

\textit{Memory offloading} strategies, such as those implemented in ZeRO-Offload \citep{rajbhandariZeROMemoryOptimizations2020}, move optimizer states and gradients to CPU memory when not actively in use, freeing up GPU memory for other operations. ZERO can also partition optimizer states and gradients across DDP processes, eliminating redundancy and significantly reducing memory footprint. \textit{Fully Sharded Data Parallel} \citep{zhaoExtendingTorchElasticStateful2020} extends this concept by sharding model parameters in addition to optimizer states and gradients.

These system-level optimizations have been instrumental in training state-of-the-art LLMs such as LLaMA3 \citep{touvronLlamaOpenFoundation2023}, GPT-3 \citep{brownLanguageModelsAre2020}, Mistral \citep{jiangMistralEfficientComposable2023}, and Gopher \citep{raeScalingLanguageModels2021} on multi-node, multi-GPU clusters.

While these distributed computing solutions enable the training of large models by leveraging extensive hardware resources, they come with increased system complexity and operational costs. Therefore, there is a pressing need for alternative approaches that reduce memory consumption without relying solely on distributed computing resources. Optimization techniques that approximate parameters or optimizer states offer a promising direction for making LLM training more accessible and efficient.

\paragraph{Parameter-Efficient Fine-Tuning}

PEFT techniques efficiently adapt pre-trained language models to various downstream applications without fine-tuning all the model's parameters \citep{dingDeltaTuningComprehensive2022}, significantly reducing the computational and memory overhead.

Among these techniques, the popular LoRA \citep{huLoRALowRankAdaptation2021} parametrizes a weight matrix $W \in \mathbb{R}^{n \times m}$ as:
\begin{equation}
 W = W_0 + BA,
\end{equation}
where $W_0$ is a frozen full-rank pre-trained weight matrix, and $B \in \mathbb{R}^{n \times r}$ and $A \in \mathbb{R}^{r \times m}$ are trainable low-rank adapters to be learned during fine-tuning. Since the rank $r \ll \min(m, n)$, the adapters $B$ and $A$ contain significantly fewer trainable parameters, reducing memory requirements for both parameter and optimizer states.

LoRA has been extensively used to reduce memory usage during fine-tuning, effectively enabling large models to be adapted to new tasks with minimal additional memory overhead. There are a few variants of LoRA proposed to enhance its performance \citep{renduchintalaTiedLoraEnhacingParameter2023, shengSLoRAServingThousands2023, zhangLORAFAMEMORYEFFICIENTLOWRANK, xiaChainLoRAEfficient2024}, supporting multi-task learning \citep{wangMultiLoRADemocratizingLoRA2023}, and further reducing the memory footprint \citep{dettmersQLoRAEfficientFinetuning2023}. Its variant, ReLoRA \citep{lialinReLoRAHighRankTraining2023}, extends LoRA's approach to pre-training by periodically updating the frozen weight matrix $W_0$ using the previously learned low-rank adapters. This incremental updating allows for continual learning without storing entire optimizer states for all parameters, leading to faster training times and lower computational costs. Furthermore, this allows for rapid adaptation of large models to multiple downstream tasks without storing separate copies of the entire model for each task.

Despite their benefits, recent works have highlighted several limitations of low-rank reparameterization approaches. LoRA does not consistently achieve performance comparable to full-rank fine-tuning, particularly in complex tasks \citep{xiaChainLoRAEfficient2024}. In pre-training from scratch, methods like ReLoRA require an initial phase of full-rank model training as a warmup before optimizing in the low-rank subspace \citep{lialinReLoRAHighRankTraining2023}. The shortcomings of low-rank parameter reparameterization suggest that alternative strategies are needed to achieve both memory efficiency and high performance.

\paragraph{Gradient Low-Rank Projection (GaLore)}

An alternative to parameter approximation is the approximation of the optimizer states. By reducing the memory footprint associated with optimizer states, it is possible to maintain full-parameter learning—thus preserving model capacity and performance—while achieving significant memory savings.

The core idea behind GaLore \citep{zhao2024galore} is to exploit the slowly changing low-rank structure of the gradient matrix $g \in \mathbb{R}^{n \times m}$, rather than approximating the weights. During neural network training, gradients naturally exhibit low-rank properties, a phenomenon studied extensively in both theoretical and practical settings \citep{zhaoZerOInitializationInitializing2022,cossonLowRankGradientDescent2023,yang2023spectral}. This intrinsic low-rank structure of gradients has been applied to reduce communication costs \citep{wangATOMOCommunicationefficientLearning,vogelsPowerGossipPracticalLowRank2020} and to decrease memory footprints during training \citep{gooneratneLowrankGradientApproximation2020,huangLowRankGradientDescent2023}.

Specifically, consider the compact SVD decomposition of the gradient matrix \(\mathbf{g} = \mathbf{P} \Sigma \mathbf{Q}^{T}\), where \(\mathbf{P} \in \mathbb{R}^{n \times r}\) and \(\mathbf{Q} \in \mathbb{R}^{m \times r}\) are the associated semi-orthognal matrices.  Then, GaLore projects the gradient matrix $\mathbf{g}$ into a low-rank form:
\begin{equation}
    \mathbf{g}_{\text{low-rank}} = \mathbf{P}^{T} \mathbf{g}.
\end{equation}
Here, $r \ll \min(n, m)$ is the target rank, \(n\) is the parameter count, \(m\) is the batch size and $\mathbf{g}_{\text{low-rank}}$ serves as an efficient approximation of the original gradient. The projection matrix $\mathbf{P}$ is updated periodically (e.g., every 200 iterations), which incurs minimal amortized computational cost.

By operating on low-rank approximations of the gradients, GaLore significantly reduces the memory footprint, leading to up to \textbf{30\%} memory reduction compared LoRA \citep{zhao2024galore}. Moreover, GaLore maintains full-parameter learning, allowing updates to all model parameters, leading to better generalization and performance than low-rank adaptation methods. Further, GaLore is agnostic to the choice of optimizer and can be easily integrated into existing optimization algorithms with minimal code modifications.

While GaLore offers significant memory savings and enables full-parameter learning, its performance has yet to match that of optimizers in full optimizer state space. Reliance on low-rank gradient approximations may not fully capture the rich optimization dynamics. These limitations suggest that while GaLore is a valuable step toward memory-efficient training, further enhancements are necessary to bridge the performance gap with standard optimizers.

\paragraph{Our Approach}

In this work, we propose to bridge the gap by incorporating a second-order regularizer into the low-rank gradient estimate, which adjusts parameter updates more effectively, leading to faster convergence. We show that applying the inverse of the empirical Fisher Information Matrix (FIM) to the low-rank gradients leads to variance reduction of the gradient estimate, incorporates information about the curvature of the loss landscape, and reduces dependence on the starting point. All of these lead to significantly faster convergence, especially in a limited iteration regime.

We introduce the \textit{\lowrank} algorithm, a matrix-free algorithm for efficiently applying the inverse FIM to the low-rank gradients, using Woodbury Identity, Cholesky Decomposition, and Matrix-Vector Products, all of which can be efficiently implemented on the GPU. Further, our approach does not require any explicit layer-wise information or significant computational overhead, as is seen in existing approaches like K-Fac \citep{martens2015optimizing}.

 We validate the effectiveness of \textit{\lowrank} through extensive empirical evaluations. Pre-training experiments on LLaMA models with 60M, 300M, and 1.1B parameters using the C4 dataset demonstrate that \textit{\lowrank} achieves significantly lower perplexity than GaLore without additional memory overhead, indicating faster convergence within the same computational budget.

 Furthermore, we showcase the practical benefits of \textit{\lowrank} in fine-tuning tasks. We fine-tune the TinyLlama 1.1B model for function calling using the TinyAgent framework. Our results show that \textit{\lowrank} significantly outperforms LoRA in this setting, achieving an accuracy of \textbf{83.09\%} on the TinyAgent dataset. This performance significantly surpasses 16-bit LoRA and exceeds that of GPT-4-turbo by 4\%, all while using \textbf{30\%} less memory.




